\documentclass[a4paper,11pt]{article}

\usepackage{latexsym}
\usepackage[empty]{fullpage}
\usepackage{titlesec}
\usepackage{marvosym}
\usepackage[usenames,dvipsnames]{color}
\usepackage{verbatim}
\usepackage{enumitem}
\usepackage[hidelinks]{hyperref}
\usepackage{fancyhdr}
\usepackage[english]{babel}
\usepackage{tabularx}
\input{glyphtounicode}

\pagestyle{fancy}
\fancyhf{} % clear all header and footer fields
\fancyfoot{}
\renewcommand{\headrulewidth}{0pt}
\renewcommand{\footrulewidth}{0pt}

% Adjust margins
\addtolength{\oddsidemargin}{-0.5in}
\addtolength{\evensidemargin}{-0.5in}
\addtolength{\textwidth}{1in}
\addtolength{\topmargin}{-.5in}
\addtolength{\textheight}{1.0in}

\urlstyle{same}

\raggedbottom
\raggedright
\setlength{\tabcolsep}{0in}

% Sections formatting
\titleformat{\section}{
  \vspace{-4pt}\scshape\raggedright\large
}{}{0em}{}[\color{black}\titlerule \vspace{-5pt}]

% Ensure that generate pdf is machine readable/ATS parsable
\pdfgentounicode=1

%-------------------------
% Custom commands
\newcommand{\resumeItem}[2]{
  \item\small{
    \textbf{#1}{: #2 \vspace{-2pt}}
  }
}

% Just in case someone needs a heading that does not need to be in a list
\newcommand{\resumeHeading}[4]{
    \begin{tabular*}{0.99\textwidth}[t]{l@{\extracolsep{\fill}}r}
      \textbf{#1} & #2 \\
      \textit{\small#3} & \textit{\small #4} \\
    \end{tabular*}\vspace{-5pt}
}

\newcommand{\resumeSubheading}[4]{
  \vspace{-1pt}\item
    \begin{tabular*}{0.97\textwidth}[t]{l@{\extracolsep{\fill}}r}
      \textbf{#1} & #2 \\
      \textit{\small#3} & \textit{\small #4} \\
    \end{tabular*}\vspace{-2pt}
}

\newcommand{\resumeSubSubheading}[2]{
    \begin{tabular*}{0.97\textwidth}{l@{\extracolsep{\fill}}r}
      \textit{\small#1} & \textit{\small #2} \\
    \end{tabular*}\vspace{-5pt}
}

\newcommand{\resumeSubItem}[2]{\resumeItem{#1}{#2}\vspace{-4pt}}

\renewcommand{\labelitemii}{$\circ$}

\newcommand{\resumeSubHeadingListStart}{\begin{itemize}[leftmargin=*]}
\newcommand{\resumeSubHeadingListEnd}{\end{itemize}}
\newcommand{\resumeItemListStart}{\begin{itemize}}
\newcommand{\resumeItemListEnd}{\end{itemize}\vspace{-5pt}}

%-------------------------------------------
%%%%%%  CV STARTS HERE  %%%%%%%%%%%%%%%%%%%%%%%%%%%%


\begin{document}

%----------HEADING-----------------
\begin{tabular*}{\textwidth}{l@{\extracolsep{\fill}}r}
  \textbf{\href{https://www.linkedin.com/in/erikboeters}{\Large Erik Boeters}} & E-mail : \href{mailto:e.boeters@gmail.com}{e.boeters@gmail.com}\\
  \href{https://www.linkedin.com/in/erikboeters}{https://www.linkedin.com/in/erikboeters} & Telefoon : +31-6-1348-1804 \\
  \href{https://github.com/boetersep}{https://github.com/boetersep} & Woonplaats : Naaldwijk \\
\end{tabular*}

%-----------EDUCATION-----------------
\section{Opleidingen}
  \resumeSubHeadingListStart
    \resumeSubheading
      {Haagse hogeschool}{Den Haag}
      {Software engineering deeltijd; Propedeuse}{2008 -- 2009}
    \resumeSubheading
      {Mondriaan onderwijsgroep}{Voorburg}
      {Technische kantoorautomatisering}{2000 -- 2004}
      \resumeItemListStart
        \resumeItem{Stageopdracht: Thales, aansturing klimaat-simulatiekast (2003)}
		  {Samen met een student embedded software engineering klimaat-simulatiekast 
		  aansturing software geschreven. Mijn rol was het ontwikkelen van de webinterface
		  waar gebruikers via een formulier een test-cyclus kunnen definieren en kunnen
		  starten. De uitkomst is een rapport met de daadwerktelijk gemeten temperatuur 
		  en relatieve luchtvochtigheid waardes gedurende de test-cyclus.
		  \par
		  Java, JSP, MySQL, Linux
		  }
		\resumeItem{Stageopdracht: 3e jaars studenten Java laten programmeren (2004)}
		  {3e jaars studenten Java bijbrengen door ze een logfile parser te laten schrijven.}
      \resumeItemListEnd
	  
  \resumeSubHeadingListEnd


%-----------EXPERIENCE-----------------
\section{Ervaring}
  \resumeSubHeadingListStart

    \resumeSubheading
      {ANWB}{Den Haag}
      {Software Engineer}{2014 -- heden}
	  \par
	  Werkzaam bij afdeling bij ANWB Online, nu IT Kanalen, bestaande uit zgn. portal, 
	  app en platform scrum teams. Als eerst 2 jaar in het "apps en mobiele website" team,
	  daarna 5 jaar in het "Vrije tijd" team, later omgedoopt tot het Fiets team. De laatste
	  2 jaar als techlead van het Fiets team. Een selectie van de activiteiten/projecten waar
	  ik aan heb gewerkt:
      \resumeItemListStart
        \resumeItem{Techlead team Fiets (2019 -- heden)}
          {Technische oplossingsrichting bepalen aan de hand van de architectuurrichtlijnen 
		  van de organisatie.
		  \par
		  Coachen van een trainee, inwerken nieuwe collega's, job interviews afnemen, workshops 
		  geven/organiseren voor andere techleads binnen de afdeling.
		  }
        \resumeItem{ANWB Fietsknooppuntenplanner (2021)}
          {De ANWB Fietsknooppuntenplanner stelt gebruikers in staat om een fietsroute te
		  plannen over het landelijke fietsknooppuntennetwerk via de website van ANWB.
		  \par
		  Ontwikkeld aan de backend en de frontend. De routing engine geconfigureerd (OSRM)
		  zodat er prioriteit aan het fietsknooppuntennetwerk wordt gegeven wanneer er een 
		  route wordt gepland. Dit project is in 2021 volledig herbouwd. In overleg met
		  collega's, management en architect heb ik er voor gekozen om de backend in
		  Node.JS in plaats van Java te implementeren zodat frontend developers er eventueel 
		  ook aan bij kunnen dragen.
		  \par
		  React, Less, Webpack, Node.JS, Apollo GraphQL, Cypress, Turf.js, GitLab, OSRM, Docker,
		  Kubernetes, Flux, AWS, Memcached, Elasticsearch
		  }
        \resumeItem{ANWB Fietslease funnel (2020 -- heden)}
          {Gebruikers in staat stellen om een e-bike te configureren en te bestellen via de 
		  website van ANWB. Verantwoordelijk voor realisatie van de API, de frontend en
		  de integratie met het CRM van ANWB. Ik onderhoud contact met de API leverancier 
		  en leden van het CRM team van ANWB.
		  \par
		  React, Less, Cypress, Webpack, Apigee.
		  }
        \resumeItem{Data importer (2019 -- heden)}
          {De Data importer is een ETL applicatie gebaseerd op Apache Camel om data
		  uit diverse bronnen in Elasticsearch te importeren. De applicatie wordt door vijf andere
		  teams binnen de afdeling gebruikt. Ik ondersteun deze teams bij het aansluiten van 
		  nieuwe bronnen en troubleshooten bij verstoringen.
		  \par
		  Java, Spring Boot, Apache Camel, Elasticsearch
		  }
        \resumeItem{Zoek \& Filter framework (2018 -- heden)}
          {Het Zoek \& Filter framework is gebouwd om met een minimale inspanning een dataset 
		  doorzoekbaar te maken via de website van ANWB. Ik ben verantwoordelijk voor het schrijven 
		  van de API's en de frontend. Ik ondersteund andere teams bij het implementeren van
		  nieuwe Zoek \& Filter instanties, het configureren van Elasticsearch en troubleshoooting.
		  \par
		  React, Less, Webpack, Cypress, Apigee, Elasticsearch
		  }
        \resumeItem{E-bike detail informatiepagina  (2017 -- 2021)}
          {Tonen van e-bike informatie, denk aan alegemene technische specificaties, een 
		  integratie met de verzekeringfunnel van ANWB, voorraadinformatie en dealerlocator.
		  \par
		  React, Less, Webpack, Cypress, Apigee, Elasticsearch
		  }
        \resumeItem{E-bike beheer tool  (2016 -- heden)}
          {Beheren van e-bike informatie ten behoeve vam de ANWB e-bike vergelijker (Zoek \& Filter)
		  applicatie en detail informatiepagina. Voorheen was dit een Java/Angular applicatie, maar
		  deze is medio 2020 omgebouwd naar een Mendix applicatie door een externe partij.
		  Kleine aanpassingen aan deze applicatie voer ik zelf uit.
		  \par
		  Java, Mendix, MongoDB}
        \resumeItem{Mobiele website (2014 -- 2016)}
          {De mobiele website van ANWB, bestaande uit het nieuws, verzekeringfunnel, 
		  verkeersinformatie, zoekpagina voor "Land van ANWB uitjes". Een Java applicatie met
		  diversie integraties naar CMS, externe API's en Elasticsearch. Deze applicatie
		  is in 2016 uitgefaseerd.
		  \par
		  jQuery, Java, Jersey}
      \resumeItemListEnd

    \resumeSubheading
      {Eleven B.V.}{Rijswijk}
      {Software Engineer}{2009 -- 2014}
	  \par
	  Webbureau voor het MKB, levert websites, webshops en webapplicaties. Samen met een collega 
	  (voormalig werkgever) ben ik hier gaan werken na de overname van zijn bedrijf, MediaActief.
	  Bij Eleven heb ik het vak geleerd, ervaring opgedaan met o.a. Java, JAX-RS, JSP, Magento 
	  en MySQL.
      \resumeItemListStart
        \resumeItem{Toolbox CMS}
          {Modulair CMS framework, wordt gebruikt voor de website (en soms interne applicaties) 
		  van klanten. Toolbox CMS is de core van het bedrijf; hier worden bijna alle projecten op
		  gebaseerd.
		  \par
		  Java, Spring, PostgreSQL, ExtJS, jQuery}
        \resumeItem{Risico-inventarisatie en -evaluatie Koninklijke Metaalunie (2011)}
          {De RI\&E Metaalbewerking is een praktisch instrument om te kunnen voldoen aan
		  de wettelijke RI\&E-verplichting. Mijn rol was het ontwikkelen van de formulieren ten 
		  behoeve van de vragenlijsten.
		  \par
		  Java, ExtJS, jQuery}
		\resumeItem{Portal Van den Bos Flowerbulbs (2009)}
		  {Mijn rol in dit project was het realiseren
		  en onderhouden van de website. In een later stadium heeft Van den Bos Flowerbulbs besloten
		  de data direct uit Navision te ontsluiten naar een portal. Voorheen werden bijvoorbeeld de 
		  bestellingen in een database opgeslagen en dit werd door een gebruiker weer in Navision
		  ingevoerd. Mijn rol was o.a. het realiseren van de koppeling tussen Navision en het portal.
          \par
		  Java, jQuery, Jasper reports}
		\resumeItem{Diverse webshops}
          {Verschillende webshops geconfigureerd en custom modules geschreven, om bijvoorbeeld op
		  maat gemaakt aluminium te bestellen en te versturen.
		  \par
		  PHP, Magento}
        \resumeItem{Dianet intranet (2013)}
          {Dianet is een kennis- en expertisecentrum op het gebied van dialyse. Mijn bijdrage aan het
		  project: verwerken van de inschrijvingen voor trainingen die via de website aangevraagd 
		  worden, hieronder valt onder andere het afrekenen via iDEAL, het afdrukken/downloaden 
		  van certificaten en het beschikbaar maken van lesmateriaal per training.
		  \par
		  Java, ExtJS, jQuery, iDEAL integratie}
        \resumeItem{Mijn Caiway (2011 -- 2014)}
          {Mijn rol als webdeveloper is het ontwikkelen van extra functionaliteiten voor Mijn Caiway. 
		  Denk aan de funcionaliteit om het voor klanten mogelijk te maken een telefoonnummer door te
		  schakelen. Het registreren en communiceren van nieuwe inschrijvingen naar het klantcontactcentrum.
		  \par
		  Java, Maven JAX-RS, JSP, jQuery}
      \resumeItemListEnd

    \resumeSubheading
      {Allround Computer Service}{Maasdijk}
      {IT Field engineer}{2005 -- 2009}
      \resumeItemListStart
        \resumeItem{Onderhoud hardware/software}
          {Hardware en software onderhoud bij zowel zakelijke en particuliere klanten. Virussen
		  verwijderen, backups inregelen, e-mail inregelen.}
        \resumeItem{Werkvoorbereiding}
          {PC's, servers en netwerkapparatuur configureren.}
        \resumeItem{Helpdesk}
          {Storingen registreren en op afstand analyseren en oplossen.}
      \resumeItemListEnd

  \resumeSubHeadingListEnd

%--------PROGRAMMING SKILLS------------
\section{Vakkennis}
  \resumeSubHeadingListStart
    \item{
      \textbf{Programmeertalen}{: Javascript (ES6), HTML, CSS, Java, PHP, Lua}
    }
    \item{
      \textbf{Certificaten}{: Professional Scrum Master I, Prince2, ITIL, Cisco CCNA, Hippo developer certificate, Elasticsearch by Trifork, Basic communication skills by ANWB, NLP by ANWB}
    }
  \resumeSubHeadingListEnd

\section{Algemeen}
  \resumeSubHeadingListStart
    \item{
      \textbf{Talen}{: Nederlands, Engels}
    }
    \item{
      \textbf{Rijbewijs}{: A, B, C}
    }
    \item{
      \textbf{Hobby's}{: software ontwikkeling, wandelen, fietsen, dagje weg in Nederland, hardlopen, tuinieren }
    }
	\item{
      \textbf{Eigenscappen}{: autodidact, analytisch, verbinder, geduldig, goed kunnen luisteren }
    }
  \resumeSubHeadingListEnd

%-------------------------------------------
\end{document}
